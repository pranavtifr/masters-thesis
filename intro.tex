\chapter{Introduction}
Though the general physical processes driving cosmological evolution and 
large scale structure formation is reasonably well understood, many of the 
details which are essential for understanding how galaxies and clusters of 
galaxies form and evolve are still unclear. One such important factor is the 
knowledge of the distribution of baryonic and dark matter in galaxies and
clusters. We know that stellar mass accounts for only approximately 10\% of the 
baryonic mass in the universe. The rest resides in diffuse components such as halos. 
\\
Historically, Diffuse components are measured via X-Ray emissions and Thermal 
Sunyaev-Zeldovich effect (tSZ). But because of the sensitivity of the experiments
just relying on these will help us detect only the hot and dense diffuse
components.  A possible way of observing these diffuse components would be to 
use gravitational lensing fields. Since these lensing fields
provide an estimate of the matter distribution in large scale structure. 
Especially, with the recent RCSLens dataset, and the ongoing KiDS dataset, weak 
lensing has become a precision tool in understanding large-scale structure, but
still the lack of understanding of baryonic physics at small scales, leads to
uncertainity in our estimates of matter distribution. \\
These missing insights from both the methods can be compensated by cross-correlating thermal SZ probes with the weak lensing fields. Since,
cross-correlations also have the advantage of being immune to systematic effects 
which doesn't correlate with the signals, It provides a powerful method 
for extracting information from these probles. 
\\
A recent attempt at cross-correlation found that, the data supports 
WMAP-7yr cosmology more than the Planck Cosmology \cite{tszrcscross}.
While, this result was found using the tSZ maps provided by the Planck
collaboration. The results beg the question of systematic deviations 
existing in the skymaps provided by the Planck collaboration. 
It is therefore, Useful to consider independent algorithms which extract 
information from the multi frequency CMB observations such as WMAP and
Planck. The methodology presented by us here is the first attempt at 
using unsupervised machine learning to perform component separation in a
model independent manner.

This work consists of 2 parts. a) We use an independent methodology to exact the tSZ maps from Planck's frequency data. This provides us with hints, incase
there exists systematic deviations from the planck skymaps. b) We compute cross-correlations between tSZ maps and Weak Lensing maps by various sky surveys.
\\
This thesis is divided as follows, We initially review the relevant physics of the data we would be using (\ref{relevantphysics}). After which, We explain our
component separation technique which also includes a quick introduction to the various machine learning aspects used in this work (\ref{fg_comp}). We then,
compute the cross-correlation with various maps and datasets. (\ref{cross_correlation}). We finally present our results and concluding remarks
( \ref{conc})











%%% Local Variables:
%%% mode: latex
%%% TeX-master: "thesis"
%%% End:
