\chapter{Introduction}
Though the general processes driving cosmological evolution and large scale structure formation is reasonably well understood,
and the model of the universe is agreed upon to be the $\Lambda$-CDM model.
There exists a disagreement among various probes on the precise values of the parameters contributing to the model. 
Therefore, It is of importance to test for any systematic biases which might exist among the existing cosmological data.
\\
SZ-Effect and Gravitational Lensing are both powerful probes to explore the baryonic content of the universe. With the existance of hubble tension, there exists
a possibility of systematic biases in our data. Cross-correlations provide a powerful tool for us to avoid these systematic biases, since the systematic biases
of these two independent probes will not be correlated to each other. 
\\
This work consists of 2 parts. a) We use an independent methodology to exact the tSZ maps from Planck's frequency data. This provides us with hints, incase
there exists systematic deviations from the planck skymaps. b) We compute cross-correlations between tSZ maps and Weak Lensing maps by various sky surveys.
\\
This thesis is divided as follows, We initially review the relevant physics of the data we would be using (\ref{relevantphysics}). After which, We explain our
component separation technique which also includes a quick introduction to the various machine learning aspects used in this work (\ref{fg_comp}). We then,
compute the cross-correlation with various maps and datasets. (\ref{cross_correlation}). We finally present our results and concluding remarks
(\ref{results}, \ref{conc})











%%% Local Variables:
%%% mode: latex
%%% TeX-master: "thesis"
%%% End:
