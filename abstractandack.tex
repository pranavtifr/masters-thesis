\begin{abstract}
The knowledge of the baryonic and non-baryonic distribution of the universe is
fundamental in understanding evolution and structure formation in the Universe.
However, a large fraction of them cannot be detected directly. One of the major
ways of indirectly detecting the baryon distribution, is cross correlation
between the thermal Sunyaev Zeldovich effect (tSZ) and Weak Gravitational Lensing.
Doing an independent analysis of this cross correlation will also be useful
in detecting any systematic errors, if any, in the existing skymaps.
We compute the tSZ skymaps using a methodology independent from the Planck collaboration's piplines,
and then cross correlate these skymaps with tangential shear to compare with the existing constrains
on halo astrophysics and cosmology \cite{tszrcscross}.
This work consists of three parts,
a) Independent generation of tSZ skymaps using Unsupervised machine learning.
b) Cross-correlating the tSZ skymaps with weak-lensing maps.
This work was done partially in collaboration with Prof. Rishi Khatri.
\end{abstract}

\begin{acknowledgements}
  I would like to thank my supervisor, Prof Subhabrata Majumdar and Dr Shadab Alam for all the discussions and help
  regarding the work on Cross-correlations. I would also like to thank Prof Rishi Khatri for his help and support on the
  generation of tSZ skymaps using machine learning. I would also like to thank TIFR for providing me with the zeal and hobbes cluster in which all the computations were made.
\end{acknowledgements}

%%% Local Variables:
%%% mode: latex
%%% TeX-master: "thesis"
%%% End:
