\chapter{Relevant Physics}
\label{relevantphysics}
In this section, We explain the relevant physics, which are related to the physics quantities we are computing.
\section{Weak Lensing}
Weak Lensing is the phenomenon when a gravitational potential distorts the shape of astrophysical sources. We know that
according to general relativity, light bends when passing though a gravitational potential. This creates an lens-like effect
which distorts the sources in the background. If the lensing is weak enough that, the images are simply distorted instead of forming
multiple images, It is known as weak lensing. 

We can represent this lensing effect by the distortion tensor ($\Psi_{ij}$).
Starting with the lensing equation $\vec{\beta} = \vec{\theta} - \vec{\alpha}$, where, $\vec{\theta}$ is the true position of the
source, $\vec{\beta}$ is the observed position of the source and $\vec{\alpha}$ is the deviation.
The deviation $\vec{alpha}$ can be derived as a function of the gravitational potential, Using the geodesic equation.
\begin{align}
  \begin{split}
    \alpha^i ( \vec{\theta}) = \pdiff{\Phi(\vec{\theta})/c^2}{\theta^i}
  \end{split}
\end{align}
Where, $\Phi$ is the projected gravitational potential, Related to the gravitational potential as,

%\comment{Write the formula for $\Phi$}
We get,

\begin{align}
  \label{derbeta}
  \begin{split}
    \frac{\partial \beta_i}{\partial \theta_j} =
    \begin{pmatrix}
      1 - \pdiff{\alpha_1}{\theta_1}& \pdiff{\alpha_1}{\theta_2} \\
       \pdiff{\alpha_2}{\theta_1}& 1 - \pdiff{\alpha_2}{\theta_2}\\
    \end{pmatrix}
  \end{split}
\end{align}

We then define the distortion tensor ($\Psi_{ij}$) as,
\begin{align}
  \begin{split}
    \pdiff{\beta_i}{\theta_j} =
    \begin{pmatrix}
      1 & 0 \\
      0 & 1\\
    \end{pmatrix}
     - \Psi_{ij}
  \end{split}
\end{align}

and then the elements of the distortion tensor\footnote{The distortion tensor is symmetric because we can write the deviation $\vec{\alpha}$ as
  a gradient of the projected gravitational potential} are defined as,

\begin{align}
  \begin{split}
    \Psi_{ij} = 
    \begin{pmatrix}
      \kappa + \gamma_1 & \gamma_2 \\
      \gamma_2 & \kappa - \gamma_1 
    \end{pmatrix}
  \end{split}
\end{align}

In the weak lensing regime, $\gamma_1 \approx e_1/2$ and $\gamma_2 \approx e_2/2$ where, $e_1$ and $e_2$ are the ellipticities of the galaxies \cite{weaklensbook}

Now, Writing the shear using the terms in equation \eqref{derbeta}, We get,
\begin{align}
  \begin{split}
    \kappa = \frac12 \left( \pdiff{\alpha_1}{\theta_1} + \pdiff{\alpha_2}{\theta_2} \right)\\
    \gamma_1 = \frac12 \left( \pdiff{\alpha_1}{\theta_1} - \pdiff{\alpha_2}{\theta_2} \right)\\
    \gamma_2 = \pdiff{\alpha_1}{\theta_2} = \pdiff{\alpha_2}{\theta_1}\\
  \end{split}
\end{align}

Now, Writing these in terms of the projected gravitational potential,
\begin{align}
  \begin{split}
  \kappa = \frac{1}{2 c^2} \left(\frac{\partial^2 \Phi}{\partial \theta_1^2}+ \frac{\partial^2 \Phi}{\partial \theta_1^2} \right)\\
  \gamma_1 =\frac{1}{2 c^2} \left(\frac{\partial^2 \Phi}{\partial \theta_1^2} - \frac{\partial^2 \Phi}{\partial \theta_1^2} \right)\\
  \gamma_2 =\frac{1}{2 c^2} \left(\frac{\partial^2 \Phi}{\partial \theta_1 \partial \theta_2} \right)\\
  \end{split}
\end{align}

In order to remove the polar dependence of these terms, We make different combinations of $\gamma_1$ and $\gamma_2$.
\begin{align}
  \begin{split}
    \gamma_T = - \gamma_1 \cos(2\phi) - \gamma_2 \sin(2 \phi)\\
    \gamma_X = - \gamma_1 \cos(2\phi) + \gamma_2 \sin(2 \phi)
  \end{split}
\end{align}
Since, For the case of spherically symmetric mass distributions, $\gamma_x$ is zero, It serves as a very useful \emph{null test}.
Any deviation from zero for $\gamma_x$ can be considered as systematic error in our calculations.





Since we have no way of knowing in advance, The intrinsic ellipticities of the galaxies to study the effect of shear, We use correlation functions
to capture the effect of shear on these galaxies.\\
The observed ellipticities of the galaxies can be considered as having 2 parts, The Shear part and the Intrinsic Part, in order to remove the contribution from
the Intrinsic part we compute correlation functions, so that the intrinsic part which is randomly oriented cancel out.

For example, For the case of Cross Correlation between $\gamma_T$ and $y$\footnote{which for our purposes now is some scalar field}, 
\begin{align}
  \begin{split}
    \langle \gamma_T y \rangle &= \langle \gamma_T^{shear} y \rangle + \langle \gamma_T^{intrinsic} y \rangle \\
    &= \langle \gamma_T^{shear} y \rangle + \cancelto{0}{\langle \gamma_T^{intrinsic} y \rangle} \\
    &= \langle \gamma_T^{shear} y \rangle
  \end{split}
\end{align}
this way, We can directly compute the effect of weak lensing directly from the ellipticities without worrying about accounting
for the intrinsic shapes of galaxies.

\section{SZ Effect}
Compton Scattering is one of the major astrophysical processes which couple matter and radiation. Sunyeav - Zel'dovich effect (SZ Effect) is one
such example of Compton Scattering at low energies, which electrons in clusters of galaxies get scattered by the Cosmic Microwave background radiation.

We would specifically be in interested in the \emph{comptonization parameter} $y$, which is a dimensionless measure of the time spend by the radiation in an
electron distribution along a particular line of sight.
\begin{align}
  \begin{split}
    y(\vec{r}) = \int n_e(\vec{r}) \sigma_T dl \frac{k_B T_e(\vec{r})}{m_e c^2}
  \end{split}
\end{align}
Where,
\begin{itemize}
  \item $n_e$ is the electron concentration
  \item $\sigma_T$ is the thompson scattering cross section\footnote{since for low energies, We can take the non relativistic approximation}.
  \item $T_e$ is the temperature of the electron cloud.
\end{itemize}

When we construct skymaps for the SZ effect, We are using the change in intensity to measure this comptonization parameter to give us insights into
the distribution of electron clouds in the universe.

Now, to use these maps to compute astrophysical or cosmological parameters, We compute the cross-correlation between the two quantities. Cross-correlation
is an incredibly powerful statistical tool since it helps us avoid systematic biases, which might be present in these two quantities individually.
Cross-correlating the quantities mean that the systematic biases won't affect our final outcome as the systematic biases from two different experiments could be
assumed to be uncorrelated with each other.


%%% Local Variables:
%%% mode: latex
%%% TeX-master: "thesis"
%%% End:
